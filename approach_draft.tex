\documentclass{article}
\begin{document}
\section{Approach}
\paragraph{Notation}
Let \{0,1\}$^n$ be the set of all n-bit binary strings. The set of all binary string is expressed as \{0,1\}$^*$.  
For a string X $\in$ \{0,1\}$^n$, |X| is its length in bits, and $\vert$ X $\vert$ $_l$ = $\lceil$$\vert$ X $\vert$/l$\rceil$ is the length of X in l-bit blocks.  Let 0$^l$ and 1$^l$ denote bit strings of all zeros and all ones. 
For a bit string X and an integer l that $\vert$ X $\vert$ $\geq$ l, msb$_l$(X) denotes the most significant l bits(left most l bits) of X and lsb$_l$(X) for least significant l bits(right most l bits) of X.
For two bit string X and Y, we denote X$\|$Y  or XY as the their concatenation. For bit string X whose length in bits is multiple of integer l, we denote X parted into l-bit sub-strings as X = (X[1]X[2]$\ldots$X[n])$_l$, where X[1], X[2], $\ldots$, X[n] $\in$ \{0,1\}$^l$.
The number of bits in a string of X is denoted as len(X).

The block cipher encryption of a string X with a secret key K is denoted as E$_K$(X).
\subsection{The MAC Scheme in Cost-Effective Tag Design}
In this article, we name the MAC scheme adopted in Cost-Effective Tag Design\cite{} CETD-MAC.  
\paragraph{Shuffle Rounds, Input Length and Tag Length}
Let E$_K$(X) be the block cipher encryption with secret key K and accepting X as input. We denote $\pi$ = E$_K$(X). Then $\pi$: \{0,1\}$^n$ $\rightarrow$
\{0,1\}$^n$.
\end{document}

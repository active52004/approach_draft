\nomenclature{MAC}{Message Authentication Code: A short message block generated
with message protected. MAC is concatenated to the rightmost of the message in
transmission}
\nomenclature{Tag}{The output short block of a MAC generation scheme}
\nomenclature{(M,T)}{Message M and its tag T}
\nomenclature{TG(M, $\ldots$)}{A MAC Scheme processing the message M with
secret information in the arguments list}
\nomenclature{VF(M,T, $\ldots$)}{Verify the message M and tag T with the secret
information in the arguments list. Return 1
if TG(M,$\ldots$) = T and 0 if not.}
\nomenclature{UF-CMA}{Unforgeability under adaptive Chosen Message Attack}
\nomenclature{Pr[A]}{The probability that event A happens}
\nomenclature{Forgery(MAC, A)}{Forgery experiment on a MAC scheme with the
adversary A. Return 1 if A succeed a forgery attack otherwise 0}
\nomenclature{Forgery$_{MAC}$}{The probability that Forgery(MAC, A)=1 for an
adversary A and a MAC scheme}
\nomenclature{PRF}{Pseudo-random Function}
\nomenclature{PRP}{Pseudo-random Permutation}
\nomenclature{Adv$^{F0}_{F1}$}{The probability for an adversary to
distinguish between two functions F0 and F1}
\nomenclature{$\oplus$}{Bitwise exclusive-xor operation(XOR)}
\nomenclature{F$_k$(M)}{Function F processing input M with secret information k}
\nomenclature{E$_k$(M)}{Encrypt input M with secret key k}
\nomenclature{Block Length}{The number of bits of a block}
\nomenclature{Len(M)}{The block length of block M}
\nomenclature{GF-mult(A,B)}{Galois Field multiplication with A and B}

